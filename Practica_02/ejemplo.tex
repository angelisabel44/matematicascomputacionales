\documentclass{article} %Tipo de documento

%Algunos paquetes nesesarios, deben de ir en preambulo las paqueterias.
\usepackage[spanish]{babel} 
\usepackage[numbers,sort&compress]{natbib} %Para la bibliografía
\usepackage{graphicx} %Para incluir figuras

\title{Reporte 1: Tarea de cosa tal} %Título del documento
\author{Angel Moreno} %Autor del documento

%Hasta aquí es el preambulo, sigue el documento
\begin{document}

\maketitle %Para crear el título del documento

\section{Introducci\'{o}n}\label{intro} %Sección del documento referenciado con 'intro'

Rollo con alguna cosa sin significado \citep{libro}.
Adem\'{a}s \citet{art} dice cosas importantes y se ven unos resultados
en la figura \ref{fig} y adem\'{a}s viene rollo en el cuadro \ref{datos}.

%Forma de incluir figuras, la figura debe estar en la misma carpeta que el archivo .tex (si se puede cambiar la ruta de donde esta la imagen)
\begin{figure}
  \centering\includegraphics[width=0.5\textwidth]{demo.png}
  \caption{Un mal ejemplo porque no vienen etiquetas en los ejes.}
  \label{fig}
\end{figure}

%Para agregar tablas.
\begin{table}
  \caption{Otro mal ejemplo pero ahora con datos.}
  \label{datos}
  \begin{center}
    \begin{tabular}{lr}
      M\'{\i}nimo & 2.00 \\
      M\'{a}ximo & 80.00 \\
      Promedio & 14.33
    \end{tabular}
  \end{center}
\end{table}

%Otra sección
\section{Conclusiones}

En la secci\'{o}n \ref{intro}, hubo puro rollo. \citep{armella2002instrumentos}

%Bibliografia, recuerda siempre que agregues un referencia nueva compilar el archivo BibTex (con F11 en Texmaker), luego el compilar el .tex
\bibliography{ejemplo}
\bibliographystyle{plainnat}

\end{document}